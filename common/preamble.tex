%%%%%%%%%%%%%%%%%%%
%% Formatting for GOST
%%%%%%%%%%%%%%%%%%%

%% Notes in pdf
\usepackage{pdfcomment}

%% Support for \mathbb and \mathcal
\DeclareMathAlphabet{\mathcal}{OMS}{cmsy}{m}{n}
\let\mathbb\relax % remove the definition by unicode-math
\DeclareMathAlphabet{\mathbb}{U}{msb}{m}{n}

%% Page numbering
\usepackage{fancyhdr}
\pagestyle{fancy}
\fancyhf{}
\renewcommand{\headrulewidth}{0pt}
%\fancyhead[C]{\thepage}
\cfoot{\thepage}

%% Enable catenation of pdfs (e.g. prpending title.pdf to rest of the text)
\usepackage{pdfpages}

%% Start page numbering from 2, due to title being the first one
%% Uncomment if title pdf is not prepended in include-before clause in yaml
%\setcounter{page}{2}

%% Double spaces
\usepackage{setspace}
%\doublespacing
\onehalfspacing

%% Indents
\usepackage{indentfirst}

%% Center headings
\usepackage[center]{titlesec}
\titlelabel{\thetitle.\quad}

%% Magic option for dots in labels
\usepackage[dotinlabels]{titletoc}

\usepackage[center,font=singlespacing]{caption}

%% Proper names
\renewcommand{\lstlistingname}{Листинг}

%% Within-section numbering
\numberwithin{table}{section}
\numberwithin{figure}{section}
\numberwithin{equation}{section}
%\usepackage{chngcntr}
%\renewcommand{\counterwithin}{\@ifstar{\@csinstar}{\@csin}}
\AtBeginDocument{\numberwithin{lstlisting}{section}}

%% Cyrillic numbering of subfigures
\usepackage{subfig}
\makeatletter
\def\cyralph#1{\expandafter\@cyralph\csname c@#1\endcsname}
\def\@cyralph#1{\ifcase #1\or а\or б\or в\or г\or д\or е\or ж\or з\or и\or к\or л\or м\or н\or о\or п\or р\or с\or т\or у\or ф\or х\or ц\or ч\or ш\or щ\or э\or ю\or я\else \@ctrerr \fi}
\renewcommand{\thesubfigure}{\cyralph{subfigure}}
\makeatother

%% Hyphenation rules
\AtBeginDocument{\selectlanguage{russian}\hyphenation{также су-пер-класс су-пер-ин-тер-фейс}}

%% Reduce indentation of bibliography citations
\setlength{\csllabelwidth}{1.5em}

\usepackage{enumitem}
\setlist[enumerate]{topsep=\parskip,itemsep=\parskip}
\setlist[itemize]{topsep=\parskip,itemsep=\parskip}

%% nested enumerate -- https://stackoverflow.com/a/24650181
\renewcommand{\labelenumi}{\arabic{enumi}.} 
\renewcommand{\labelenumii}{\arabic{enumi}.\arabic{enumii}}
\renewcommand{\labelenumiii}{\arabic{enumi}.\arabic{enumii}.\arabic{enumii}}

%\AtBeginDocument{
%    \abovedisplayskip=2\parskip
%    \belowdisplayskip=2\parskip
%    \abovedisplayshortskip=2\parskip
%    \belowdisplayshortskip=2\parskip
%}

\usepackage{lscape}

%% Define \ceil*{} -- https://tex.stackexchange.com/questions/42271/floor-and-ceiling-functions/42274
\usepackage{mathtools}
\DeclarePairedDelimiter\ceil{\lceil}{\rceil}

%% Fix overset equals sign alignment in multiline equations -- https://tex.stackexchange.com/questions/257529/overset-and-align-environment-how-to-get-correct-alignment
\usepackage{aligned-overset}

%%%%
%% Threorm, algorithm and other stuff
%%%%

\usepackage{amsthm}

\newtheorem{algorithmbase}{Алгоритм}[section]
\newtheorem{theorem}{Теорема}[section]
\newtheorem{lemma}[theorem]{Лемма}
\newtheorem{statement}[theorem]{Утверждение}
\newtheorem{corollary}[theorem]{Следствие}

%\newcounter{alg}
%\numberwithin{alg}{section}

\newenvironment{algorithm}[1]{
    \begin{algorithmbase}\label{#1}
} {
    \qed
    \end{algorithmbase}
}

%\newcounter{stmt}
%\numberwithin{stmt}{section}
%
%\newenvironment{statement}[1]{
%    \newcommand{\statementproof}{
%        \textit{Доказательство.\ }%
%    }
%
%    \begin{samepage}
%    \refstepcounter{stmt}\label{#1}
%    \textbf{Утверждение \thestmt.\ }%
%} {
%    \qed
%    \end{samepage}
%}

%%%%%%%%%%%

%%%%%%%%%%%%%%%%%%%
%% Listings
%%%%%%%%%%%%%%%%%%%

%\usepackage{xcolor}   % for \textcolor

%% Fix for "Missing number, treated as zero spaces"
%% when using consecutive spaces
\newsavebox\grayarrow
\sbox\grayarrow{\raisebox{0ex}[0ex][0ex]{\ensuremath{\hookrightarrow\space}}}

\lstdefinestyle{default}{
    language     = scala,
    basicstyle   = {\linespread{1}\small\ttfamily},
    %% Uncomment to move caption below listing
    %captionpos   = b,
    breaklines   = true,
    numbersep    = 5pt,
    escapeinside = {\$\$},
    columns      = space-flexible,
    postbreak    = \usebox\grayarrow,
    showstringspaces = false,
    texcl        = true,
    %% Examples of custom keyword highlighting
    emph         = [1]{ordered,by,break,dom,call,with,and,or,not},
    emphstyle    = [1]{\bfseries},
    emph         = [2]{super,final,protected,private,package},
    emphstyle    = [2]{},
    emph         = [3]{find,resolve,defined,undefined,address,superclass,array,of,in},
    emphstyle    = [3]{\itshape},
    xleftmargin  = 2\parindent,
    literate={-}{-}1
        {=}{=}1
        {!=}{\neq}1
        {>=}{\geq}1
        {<=}{\leq}1
        {:=}{\gets}1
        {<-}{\in}1
        {->}{\rightarrow}2
        {=>}{\Rightarrow}2,
}

\lstset{style=default}

%%%%%%%
%% Allow skipping line numbers -- https://tex.stackexchange.com/a/215752
\let\origthelstnumber\thelstnumber
\makeatletter
\newcommand*\Suppressnumber{%
  \lst@AddToHook{OnNewLine}{%
    \let\thelstnumber\relax%
     \advance\c@lstnumber-\@ne\relax%
    }%
}

\newcommand*\Reactivatenumber[1]{%
  \lst@AddToHook{OnNewLine}{%
   \let\thelstnumber\origthelstnumber%
   %\setcounter{lstnumber}{\numexpr#1-1\relax}%
   \advance\c@lstnumber\@ne\relax%
  }%
}

\makeatother
%%%%%%%


%%%%%%%%%%%%%%%%%%%
%% Other
%%%%%%%%%%%%%%%%%%%

%% Workaround for pandoc-crossref issue - https://github.com/lierdakil/pandoc-crossref/issues/326
\makeatletter
\@ifpackageloaded{subfig}{}{\usepackage{subfig}}
\@ifpackageloaded{caption}{}{\usepackage{caption}}
\captionsetup[subfloat]{margin=0.5em}
%\AtBeginDocument{%
%\renewcommand*\figurename{Figure}
%\renewcommand*\tablename{Table}
%}
%\AtBeginDocument{%
%\renewcommand*\listfigurename{List of Figures}
%\renewcommand*\listtablename{List of Tables}
%}
\newcounter{pandoccrossref@subfigures@footnote@counter}
\newenvironment{pandoccrossrefsubfigures}{%
\setcounter{pandoccrossref@subfigures@footnote@counter}{0}
\begin{figure}\centering%
\gdef\global@pandoccrossref@subfigures@footnotes{}%
\DeclareRobustCommand{\footnote}[1]{\footnotemark%
\stepcounter{pandoccrossref@subfigures@footnote@counter}%
\ifx\global@pandoccrossref@subfigures@footnotes\empty%
\gdef\global@pandoccrossref@subfigures@footnotes{{##1}}%
\else%
\g@addto@macro\global@pandoccrossref@subfigures@footnotes{, {##1}}%
\fi}}%
{\end{figure}%
\addtocounter{footnote}{-\value{pandoccrossref@subfigures@footnote@counter}}
\@for\f:=\global@pandoccrossref@subfigures@footnotes\do{\stepcounter{footnote}\footnotetext{\f}}%
\gdef\global@pandoccrossref@subfigures@footnotes{}}
\newcommand*\listoflistings\lstlistoflistings
%\AtBeginDocument{%
%\renewcommand*{\lstlistlistingname}{List of Listings}
%}
\makeatother

\usepackage{mdframed}

%% Provides BVerbatim environment which boxes verbatim
%% so it can be centered -- https://tex.stackexchange.com/a/122197
\usepackage{fancyvrb}

%% Set numbered footnotes in minipages
\renewcommand{\thempfootnote}{\arabic{mpfootnote}}

%% Do not number tables
%\captionsetup[table]{labelformat=empty}

%% Enable strikethrough syntax \st of newer pandoc
\usepackage{soul}

\usepackage{csquotes}

%% Normal subscripts in listings using \textsubscript{i} -- https://tex.stackexchange.com/questions/63845/boldface-and-subscripts-in-verbatim-mode    
\usepackage{fixltx2e}

%% Hack for using latex envs inside markdown -- https://github.com/jgm/pandoc/issues/3145#issuecomment-302787889
\newcommand{\hideFromPandoc}[1]{#1}
\hideFromPandoc{
  \let\Begin\begin
  \let\End\end
}

%% TODO command
\newcommand\todo[1]{\textcolor{red}{#1}}

%%%%%%%%%%%%%%%%%%%
%% Tikz stuff
%%%%%%%%%%%%%%%%%%%

\usepackage{tikz}
\usetikzlibrary{matrix,calc,positioning,fit,graphs,arrows.meta,backgrounds,decorations.pathreplacing}

\tikzset{>=latex}

%% Align text vertically with subscript
%\tikzset{text depth=.25ex}
%% Align text vertically without subscript
\tikzset{text depth=0}

\tikzset{node/.style={minimum width=2em, minimum height=2em,draw,font=\small\ttfamily}}
\tikzset{class/.style={node,rectangle}}
\tikzset{interface/.style={node,circle}}

\def\structnodewidth{4em}
\def\structnodeheight{4ex}

\tikzset{struct base/.style={minimum width=\structnodewidth, minimum height=\structnodeheight, inner sep=0}}
% This is normal for courier:
%\tikzset{head/.style={struct base,font=\small\ttfamily\bfseries,fill=gray!20}}
%\tikzset{field/.style={struct base,font=\small\ttfamily}}
% This looks better for dejavu
\tikzset{head/.style={struct base,font=\small\ttfamily\bfseries,fill=gray!20}}
\tikzset{field/.style={struct base,font=\small\ttfamily}}

\tikzset{muted/.style={font=\footnotesize\ttfamily,text=gray}}



\newenvironment{struct}[2][]{
    \begin{scope}[node distance=0]

    \coordinate [#1] (#2) {};
    \def\headnode{#2}
    \def\lastnode{#2}

    \newcommand{\header}[2]{
        \node [head] (##1) [below=of \lastnode] {##2};
        \draw (##1.south west) -- (##1.south east);
        \def\lastnode{##1}
    }

    \newcommand{\field}[4][draw=none]{
        \node [field] (##3) [below=of \lastnode] {##4};
        \draw [##1] (##3.south west) -- (##3.south east);
        \node [muted,left=0.5em of ##3,fill=white,inner sep=0] (##3 @extra) {##2};
        \def\lastnode{##3}
    }
} {
    \node [draw,inner sep=0,fit=(\headnode) (\lastnode)] {};
    \end{scope}
}

\newcommand{\connect}[4][0.5]{
    \draw [#4] [->] (#2) -| ($ (#2) !#1! (#3) $) |- (#3);
}

\newcommand{\imtR}[3]{
    \draw [decorate,decoration={brace,amplitude=0.35em}] (#1.north east) -- (#2.south east) node [midway,font=\footnotesize\ttfamily\bfseries,right=0.2em] {#3};
}

\newcommand{\imtL}[3]{
    \draw [decorate,decoration={brace,amplitude=0.35em,mirror}] (#1.north west) -- (#2.south west) node [midway,font=\footnotesize\ttfamily\bfseries,left=0.2em] {#3};
}

%% Tikz figure
\newenvironment{tikzfigure}[3]{
    \def\figlabel{#1}
    \def\figcaption{#2}

    \begin{figure}
    \centering
    \begin{tikzpicture}[#3]
} {
    \end{tikzpicture}
    \caption{\figcaption}
    \label{\figlabel}
    \end{figure}
}
\newcommand{\tikzsubfloat}[3]{
    \subfloat[#2]{
    \begin{tikzpicture}[baseline=(current bounding box.north)]
    #3
    \end{tikzpicture}
    \label{#1}
    }
}



%%%%%%%%%%%%%%%%%%%
%% pgfplots stuff
%%%%%%%%%%%%%%%%%%%

\usepackage{pgfplots}
\usepackage{pgfplotstable}

\definecolor{googleblue}{HTML}{4285F4}
\definecolor{googlered}{HTML}{EA4335}
\definecolor{googleyellow}{HTML}{FBBC04}
\definecolor{googlegreen}{HTML}{34A753}
\definecolor{googleorange}{HTML}{FE6D00}

